
\documentclass[../portafolio.tex]{subfiles}

\begin{document}

\section{Introducción}
%\addcontentsline{toc}{chapter}{Introducción}
%\markboth{Introducción}{Introducción}

La Termodinámica es una disciplina que estudia las transformaciones de energía en forma de calor y trabajo de los sistemas macroscópicos y nos proporciona relaciones entre las propiedades físicas de un sistema, tales como la variación de temperatura, presión y volumen las cuales veremos su comportamiento a lo largo de este informe, en el cual queremos comprobar las leyes de gases ideales a través de simulaciones. Primero definiremos algunas variables, definiciones, leyes y ecuaciones necesarias para la compresión del trabajo, luego describiremos el procedimiento realizado a través de mediciones las cuales serán explayadas en este informe, junto con sus respectivos resultados obtenidos. Para así analizar el comportamiento de dichos datos por medio de las habilidades y herramientas otorgadas por la matemática, para luego concluir si es que se cumple o no la ley de los gases ideales.
%\medskip

\end{document}
