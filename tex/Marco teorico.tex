\documentclass[../portafolio.tex]{subfiles}
\begin{document}
\section{Marco Teórico}

A continuación definiremos algunas propiedades, conceptos y variables necesarias para el análisis de los resultados de los experimentos a realizar en el simulador online.
\begin{itemize}
\item \textbf{Gas Ideal:} Un gas ideas es una simplificación de los gases reales usada para hacer su estudio mas sencillo. Es en sí un gas hipotético que cumple las siguientes características.
    \begin{itemize}
    \item Las colisiones son elásticas, conservando el momento y la energía cinética
    \item La energía cinética es directamente proprcional a la temperatura
    \item Está formado por \textit{n} partículas puntuales
    \item No se tienen en cuenta las interacciones electromagnéticas de atracción y repulsión
    \end{itemize}

	\item \textbf{Presión(P):} Se define como la fuerza(F) por unidad de área(A) que actúa perpendicularmente sobre un superficie. Matemáticamente se define como  $ P= \frac{F}{A} $. En el Sistema Internacional de Unidades(S.I.) se mide en Pascal(Pa), esta es equivalente a $\frac{N}{m^{2}}$.
	\item \textbf{Temperatura(T):} Se define como la magnitud de la energía cinética de un sistema termodinámico. En el S.I. se mide en Kelvin(K°)
        \item \textbf{Volumen(V):} Se define como la extensión en tres dimensiones de una región o un sistema en el espacio. En el S.I. se mide en metros cúbicos(m³)
\end{itemize}
\subsection*{Ecuación de los Gases Ideales}
%Las hipótesis necesarias para la validez de esta escuación son:

Su expresión matemática es,
\begin{equation}
PV=nRT
\end{equation}
Donde están involucrados los términos de, Presión($P$), Volumen($V$), número de moles($n$),constante de los Gases Ideales($R$) y la Temperatura($T$).

Físicamente podemos interpretar que cuando en un proceso termodinámico el volumen, la presión o la temperatura se mantienen constantes, las otras dos propiedades presentaran una relación de proporcionalidad, ya sea directa o inversa. 

\end{document}
