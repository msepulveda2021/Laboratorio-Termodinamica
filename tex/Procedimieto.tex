\documentclass[../portafolio.tex]{subfiles}
\begin{document}


\section{Procedimiento}
\begin{enumerate}
    \item \textbf{Primera Simulación:} En la primera simulación trabajamos a una temperatura constante de 300 \textit{K°}  y con un número de n=50 partículas pesadas. Ya con esto, variamos el ancho del recipiente, partiendo en 15\textit{nm}, 13\textit{nm}, 11\textit{nm}, 9\textit{nm}, 7\textit{nm} y 5\textit{nm}. Aquí registramos los valores de la presión obtenidos para cada ancho. Luego repetimos este proceso usando un número de n=50, n=100 y n=150 de partículas pesadas y ligeras y para temperaturas de 300\textit{K°} y 600\textit{K°}.
    \item \textbf{Segunda Simulación:} Para la segunda simulación trabajamos con un número de n=50 partículas pesadas, a una presión constante de 5.9\textit{atm} y con valores de temperatura y ancho iniciales de 300\textit{K°} y 10\textit{nm} respectivamente. A continuación variamos la Temperatura partiendo en 150\textit{K°}, 225\textit{K°}, 375\textit{K°} y 450\textit{K°}. Aquí registramos el ancho para cada valor de la temperatura correspondiente.
    \item \textbf{Tercera simulación:} Para la tercera simulación repetimos el mismo procedimiento que en la segunda, pero esta vez usando un número de n=150 de partículas pesadas y a una presión constante de 17.5\textit{atm}
    \item \textbf{Cuarta Simulación:} Al igual que en la tercera simulación, aca repetimos el procedimiento de la segunda simulación, esta vez usando un número de n=250 partículas pesadas y a una presión constante de  29.2\textit{atm}.
\end{enumerate}
\end{document}
