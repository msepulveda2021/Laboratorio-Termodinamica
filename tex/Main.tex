\documentclass[twoside]{report}
\usepackage[table]{xcolor}
\usepackage[paper=letterpaper,left=3cm,right=3cm,top=3cm,bottom=2cm,includefoot]{geometry}
\usepackage{fancyhdr}
\usepackage{tabularx}
\usepackage{multirow}
\usepackage[colorlinks]{hyperref}
\usepackage{hhline}
\usepackage{amsmath}
\usepackage{enumitem}
\setenumerate{itemsep=-3pt,topsep=3pt}
\setdescription{itemsep=-3pt,topsep=3pt,leftmargin=!,labelwidth=4.5cm}
\usepackage{pgfgantt}
\usepackage{graphicx}   \graphicspath{{./img/} {./tex/img/}}
\usepackage[spanish]{babel}
\usepackage[utf8]{inputenc}
\usepackage[T1]{fontenc}
\usepackage[square,numbers,sort&compress]{natbib}
\bibliographystyle{apsrev4-1}
\usepackage{subfiles}

%\usepackage{slashbox}
\usepackage{diagbox}

\pagestyle{fancy}
\renewcommand{\footrulewidth}{0.4pt}
\renewcommand{\headrulewidth}{0.4pt}
\fancyfoot{}
%\fancyfoot[RE,RO]{\thepage}
%\fancyhead[LO,LE]{\textcolor[RGB]{127,127,127}{Portafolio - Física Computacional II (2022)}} 

\definecolor{tcc}{RGB}{217,217,217} % Table cell color

\renewcommand\tabularxcolumn[1]{m{#1}}
\setlength{\arrayrulewidth}{0.5pt}
\renewcommand{\arraystretch}{2}

\begin{document}
\subfile{portada.tex}
\tableofcontents

\subfile{introducción.tex}

\subfile{Objetivos.tex}

\subfile{marcoteorico.tex}

\subfile{materiales.tex}

\begin{figure}[ht]
    \centering
    \includegraphics[width=0.5\textwidth]{Materiales.png}
    \caption{Simulador online}
    \label{fig:simulador}
\end{figure}

\subfile{Procedimiento.tex}

\subfile{Resultados.tex}

\subfile{Analisis.tex}

\subfile{Conclusion.tex}

\bibliographystyle{amsplain}
\bibliography{referencias}
\end{document}
