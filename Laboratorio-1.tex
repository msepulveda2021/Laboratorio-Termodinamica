\documentclass[twoside]{report}

\usepackage[table]{xcolor}
\usepackage[table]{xcolor}
\usepackage[paper=letterpaper,left=3cm,right=3cm,top=3cm,bottom=2cm,includefoot]{geometry}
\usepackage{fancyhdr}
\usepackage{tabularx}
\usepackage{multirow}
\usepackage[colorlinks]{hyperref}
\usepackage{hhline}
\usepackage{amsmath}
\usepackage{enumitem}
\setenumerate{itemsep=-3pt,topsep=3pt}
\setdescription{itemsep=-3pt,topsep=3pt,leftmargin=!,labelwidth=4.5cm}
\usepackage{pgfgantt}
\usepackage{graphicx}   \graphicspath{{./img/} {./tex/img/}}
\usepackage[spanish]{babel}
\usepackage[utf8]{inputenc}
\usepackage[T1]{fontenc}
\usepackage[square,numbers,sort&compress]{natbib}
\bibliographystyle{apsrev4-1}


\begin{document}
    \fancypagestyle{frontpage}{ \fancyhf{} \renewcommand{\headrulewidth}{0pt}
    \renewcommand{\footrulewidth}{0pt}\vspace*{1\baselineskip}}

\begin{titlepage}
  \newgeometry{top=4cm, bottom=4cm, left=4cm, right=4cm} 
  \thispagestyle{frontpage}
 \begin{center}
     \includegraphics[width=0.3\textwidth]{escudo_udec.png}
     
     \vspace*{3\baselineskip}
  \textsc{\Huge \textbf{Laboratorio 1}}
  
     \vspace*{1.5\baselineskip}
  \textsc{\Huge Física IV: Termodinámica}\\

     \vspace*{4\baselineskip}
  \Large{\textbf{Florencia Andrea Fuentes Jara, Cs. Físicas\\ Martín Alberto Garrido Briano, Cs. Físicas \\ Martín Andrés Sepúlveda Zúñiga, Cs. Físicas}}\\ 
    \vspace{2\baselineskip}
 \Large{\textbf{Profesor Guía:} Dr. Claudio Alonso Faundez Araya}\\
     \vspace{1\baselineskip}
     \begin{center}
        \justify\Large{\textbf{Ayudantes:} 
        
    Arelly Daniela Núñez Vásquez  \\ Annais Belén Molina Parra \\  Renata Valentina Hernandez Lopez \\ Anahis Alexsandra Verena De La Barra Manzano}
    \vspace{2\baselineskip} 
     \end{center}
 
    
         12 de Septiembre del 2022 \\
            Concepción, Chile 
\end{center}
  
  \vspace*{4\baselineskip}
  
\end{titlepage}
 
 \chapter*{Introducción}
%\addcontentsline{toc}{chapter}{Introducción}
%\markboth{Introducción}{Introducción}
bla bla

\medskip
bla bla


\section{Marco Teórico}

A continuación definiremos algunas propiedades, conceptos y variables necesarias para el análisis de los resultados de los experimentos a realizar en el simulador online.

\textbf{Gas Ideal:} Un gas ideas es una simplificación de los gases reales usada para hacer su estudio mas sencillo. Es en sí un gas hipotético que cumple las siguientes características.
    \begin{itemize}
    \item Las colisiones son elásticas, conservando el momento y la energía cinética
    \item La energía cinética es directamente proprcional a la temperatura
    \item Está formado por \textit{n} partículas puntuales (\textit{n} \in \mathbb N)
    \item No se tienen en cuenta las interacciones electromagnéticas de atracción y repulsión
\end{itemize}
\begin{itemize}
	\item \textbf{Presión(P):} Se define como la fuerza(F) por unidad de área(A) que actúa perpendicularmente sobre un superficie. Matemáticamente se define como  $ P= \frac{F}{A} $. En el Sistema Internacional de Unidades(S.I.) se mide en Pascal(Pa), esta es equivalente a $\frac{N}{m^{2}}$.
	\item \textbf{Temperatura(T):} Se define como la magnitud de la energía cinética de un sistema termodinámico. En el S.I. se mide en Kelvin(K°)
        \item \textbf{Volumen(V):} Se define como la extensión en tres dimensiones de una región o un sistema en el espacio. En el S.I. se mide en metros cúbicos(m³)
\end{itemize}
\subsection*{Ecuación de los Gases Ideales}
Las hipótesis necesarias para la validez de esta escuación son:
\begin{itemize}
	\item 
\end{itemize}
Su expresión matemática es,
\begin{equation}
PV=nRT
\end{equation}
donde están involucrados los términos de, Presión($P$), Volumen($V$), número de moles($n$),constante de los Gases Ideales($R$) y la Temperatura($T$).

Físicamente esta establece relaciones de proporcionalidad directa e indirecta entre la Presión, la Temperatura y el Volumen.










\medskip
bla bla

\medskip
bla bla

\medskip
bla bla
\medskip

\end{document}
    
https://www.overleaf.com/5729473518gccsphxytgzv
